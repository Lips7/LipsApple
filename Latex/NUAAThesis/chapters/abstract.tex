% !TeX root = ../main.tex

\nuaasetup{
    keywords = {
            股票预测,支持向量机,超参数寻优
        },
    keywords* = {
            Stock Prediction, Support Vector Machine, Hyper-Parameter Optimaztion
        },
}

\begin{abstract}
    自改革开放以来,随着中国经济的腾飞,人们拥有的财富越来越多,而人的欲望是无止境的,
    总想通过不同的投资手段来扩大自己的财富,这便催生了各种各样的投资手段。然而,传统
    的投资手段,如基金、定期储蓄等,风险极小但利率较低,普通人难以通过这种方式来抵抗
    通货膨胀引起的货币贬值。随后,金融危机的出现导致大量资金涌入房地产,投资房地产也
    成为了大多数中产阶级的选择,但近几年来,国家出台了一系列政策来遏制房地产泡沫,延
    缓房价上升。于是,更多资金涌入了二级市场,投资者们希望通过股票投资来迅速实现资产
    增值,而高收益往往伴随着高风险,如何准确预测股价的走势便成了当下股市研究的一个重
    要课题。本文从3600余种股票中,选取出两种特定的模式,以此数据集为基础,来对股票未
    来多个交易日或交易周的股票最低价平均值进行预测。

    本文第一章介绍了研究背景及意义,国内外股市的研究现状;第二章介绍了将使用的机器学
    习理论;第三章为模型的实现过程,使用标准化方法预处理数据的12个指标,将这些指标作
    为支持向量机模型的输入向量,选择高斯核函数,利用TPE算法与交叉验证方法来对参数进
    行自动寻优,得到最佳的核参数和一个回归模型。最后在测试集上进行验证,得到多个周期
    的股价预测结果;第四章为实验结果的总结与分析。

\end{abstract}

\begin{enabstract}
    Since the Reform and Opening up, as China's economy has taken off,
    people have more and more wealth, and desires to expand their wealth
    through different means of investment, is endless, which has spawned
    various means of investment. While the traditional investment instruments,
    such as funds, regular saving, etc., have minimal risk but low interest rates.
    making it difficult for the ordinary people to resist the currency depreciation
    caused by inflation in this way. Subsequently, the financial crisis led to an
    influx of money pouring into the real estate, investment in which
    has become the choice of most the middle class. However, in recent years,
    China has introduced a series of policies to curb the real estate
    bubble and slow down the rise of housing prices. As a result, more money
    poured into the secondary market, investors hope to achieve asset
    appreciation quickly through stock investment. While high returns are
    often accompanied by high risk, how to accurately predict the trend
    of stock prices has become an important topic in the current stock market
    research. In this paper, two specific patterns are selected from more
    than 3600 stocks. Based on this data set, the average value of lowest
    prices for multiple trading days or trading weeks in the future is predicted.

    The first chapter introduces the research background and significance,
    the research status of stock market at home and abroad. The second chapter
    introduces the machine learning theory to be used. The third chapter is the
    application of the model, standardized method is applied to preprocess
    the twelve indexes of the data, which is used as the input vectors of the support
    vector machine model. Finally, it is verified on the test set, multiple
    stock price prediction results are obtained. The fourth chapter is the summary 
    of the experiment.

\end{enabstract}
