% !TeX root = ../main.tex

\nuaasetup{
    keywords = {
            股票预测,支持向量机,超参数寻优
        },
    keywords* = {
            Stock Prediction, Support Vector Machine, Hyper-Parameter Optimaztion
        },
}

\begin{abstract}
    自改革开放以来,人们拥有的金钱越来越多,并且总想通过不同的方式来扩大自己的财富,
    这便催生了各种各样的投资手段。然而传统的投资手段,如基金、定期储蓄等,风险极小但
    利率较低,难以抵抗通货膨胀。随后金融危机的出现导致房地产投资成为了一种新选择,但
    近几年国家出台了一系列政策来延缓房价上升。于是,许多资金涌入了股票二级市场,投资
    者们希望通过这种方式来实现资产的迅速增值,而高收益往往伴随着高风险,如何准确预测
    股价的走势便成了当下股市研究的一个重要课题。

    目前比较热门的量化投资的方法,即利用机器学习手段对股价进行预测,往往仅针对一种或
    多种股票,研究不同的数值指标对预测效果的影响,或者从多种股票中选取收益率较高的股
    票。不同于这些方法,本文针对中国大陆所有股票,在日线和周线分别选取的两种特定数据
    形态组合,利用支持向量机对未来多个交易周期的股票最低价平均值进行预测,并给出相应
    的投资建议。本文的主要内容如下:第一章介绍了本文的研究背景及意义,和有关股票预测
    的国内外研究现状;第二章介绍了支持向量机的理论基础;第三章对模型的实现过程做了详
    细的解释,包括选取指标,标准化数据,和选取核函数以及各种超参数寻优方法的对比,还
    有对实验效果的评价指标;第四章呈现了本文的实验结果,并对得到的结果数据进行了分析
    和总结,从中挖掘数据的价值,最后给出相应的投资决策建议;第五章为本文工作的总结以
    及对未来的展望。

    本文经实验研究后发现,支持向量机回归模型对特定数据形态组合的股票有不错的预测效果
    ,不仅体现在短期预测上,其甚至在一些中长期预测上也有相当好的效果。
    
\end{abstract}

\begin{enabstract}
    Since the Reform and Opening up, people have more and more money, 
    and desire to expand their wealth through different means, which 
    has spawned various methods of investment. While the traditional 
    investment instruments such as funds, regular saving, etc., have 
    minimal risk but low interest rates, making it difficult to resist 
    the inflation. Subsequently, the financial crisis led to investing
    on real estate has become a new choice. However, in recent years, 
    China has introduced a series of policies to slow down the rise 
    of housing prices. As a result, more money poured into the secondary 
    market, investors hope to achieve asset appreciation quickly through 
    stock investment. While high returns are often accompanied by high 
    risks, how to accurately predict the trend of stock prices has become 
    an important topic in the current stock market research. 

    The more popular current approach to quantitative investing, which 
    uses machine learning to make predictions about stock prices, tends 
    to target only one or more stocks, studying the effect of different 
    numerical indicators on the predictive effect, or picking stocks 
    with higher yields from a wide range of stocks. In contrast to the 
    traditional machine learning approach to stock prediction, this paper 
    used a support vector machine to predict the average of stock lows 
    over multiple trading cycles in the future, using all stocks in China 
    mainland for two specific data patterns selected on the daily and 
    weekly line, and gives investment recommendations. The main contents 
    of this paper are as follows: Chapter 1 introduces the background 
    and significance and the current status of domestic and international 
    research on stock prediction; Chapter 2 introduces the theoretical 
    basis of the support vector machine; Chapter 3 provides a detailed 
    explanation of the implementation process of the model, including 
    the selection of indicators, standardized data, the selection of 
    kernal functions and the difference between hyperparameter optimization 
    methods, and the evaluation of the experimental result; Chapter 4 
    presents the results of this paper, analyzes and summarizes the data, 
    extracts the value, and finally gives the investment decision 
    recommendations; Chapter 5 provides a summary of the work of this 
    paper and a vision for the future.

    In this paper, it is found that the support vector machine regression 
    model has a good predictive effect on stocks with a particular combination 
    of data patterns, not only on short-term predictions, but also on some 
    medium and long-term predictions.

\end{enabstract}
