% !TeX root = ../main.tex

\nuaasetup{
    keywords = {
            股票预测,支持向量机,超参数寻优
        },
    keywords* = {
            Stock Prediction, Support Vector Machine, Hyper-Parameter Optimaztion
        },
}

\begin{abstract}
    自改革开放以来,随着中国经济的腾飞,人们拥有的财富越来越多,并且总想通过不同的投
    资手段来扩大自己的财富,这便催生了各种各样的投资手段。然而,传统的投资手段,如基
    金、定期储蓄等,风险极小但利率较低,普通人难以通过这种方式来抵抗通货膨胀引起的货
    币贬值。随后,金融危机的出现导致大量资金涌入房地产,投资房地产也成为了大多数中产
    阶级的选择,但近几年来,国家出台了一系列政策来遏制房地产泡沫,延缓房价上升。于是
    ,更多资金涌入了二级市场,投资者们希望通过股票投资来迅速实现资产增值,而高收益往
    往伴随着高风险,如何准确预测股价的走势便成了当下股市研究的一个重要课题。相比于传
    统的利用机器学习进行股票预测的方法,本文针对中国大陆所有股票,在日线和周线所分别
    选取的两种特定数据形态组合,利用支持向量机对未来多个交易周期的股票最低价平均值进
    行预测,并给出相应的投资建议。

    本文的主要内容如下:第一章介绍了本文的研究背景及意义,和有关股票预测的国内外研究
    现状;第二章介绍了支持向量机的理论基础;第三章对模型的实现过程做了详细的解释,包
    括选取指标,标准化数据,和选取核函数以及超参数寻优方法,以及对实验效果的评价指标
    ;第四章呈现了本文的实验结果,并对得到的结果数据进行了分析和总结,最后给出相应的
    投资决策建议。

\end{abstract}

\begin{enabstract}
    Since the Reform and Opening up, as China's economy has taken off,
    people have more and more wealth, and desire to expand their wealth
    through different means of investment, which has spawned various means
    of investment. While the traditional investment instruments, such as
    funds, regular saving, etc., have minimal risk but low interest rates.
    making it difficult for the ordinary people to resist the currency
    depreciation caused by inflation in this way. Subsequently, the financial
    crisis led to an influx of money pouring into the real estate, investment
    in which has become the choice of most the middle class. However, in
    recent years, China has introduced a series of policies to curb the real
    estate bubble and slow down the rise of housing prices. As a result,
    more money poured into the secondary market, investors hope to achieve
    asset appreciation quickly through stock investment. While high returns are
    often accompanied by high risk, how to accurately predict the trend of
    stock prices has become an important topic in the current stock market
    research. In contrast to the traditional machine learning approach to 
    stock prediction, this paper used a support vector machine to predict 
    the average of stock lows over multiple trading cycles in the future, 
    using all stocks in Chine mainland for two specific data patterns selected 
    on the daily and weekly line, and gives investment recommendations.

    The main contents of this paper are as follows: Chapter 1 introduces 
    the background and significance and the current status of domestic and 
    international research on stock prediction; Chapter 2 introduces the 
    theoretical basis of the support vector machine; Chapter 3 provides a 
    detailed explanation of the implementation process of the model, including 
    the selection of indicators, standardized data, the selection of kernal 
    functions and hyperparameter optimization methods, and the evaluation of 
    the experimental result; Chapter 4 presents the results of this paper, 
    analyzes and summarizes the data, and finally gives the investment decision 
    recommendations.

\end{enabstract}
