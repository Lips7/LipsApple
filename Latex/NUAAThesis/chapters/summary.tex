% !TeX root = ../main.tex
\chapter{总结与展望}

\section{总结}

SVM算法是基于统计学习理论和结构风险最小化原则的新兴机器学习方法,与神经网络等传统方法相比,SVM能够较好地解决局部极值和维度灾难的问题,
这使得SVM被应用在了经济、社会等多个领域,并取得了很好的效果。在股票市场上,量化投资的兴起带动了机器学习在股价预测方面的应用。

本文在前人的基础上,以中国大陆股票市场作为研究对象,利用SVM建立了针对多个周期下股票最低价平均值的预测模型。
取2016年1月04日至2019年8月15日之间股票所有交易日的数据,分别在日线和周线上筛选出2种特定的数据形态组合,
利用$Z\text{-}score$方法标准化数据,使用交叉验证法与TPE算法相结合进行超参数寻优得到SVM回归预测模型,
应用在测试数据上,获得了较低的预测误差。可以说明,在特定的数据形态组合出现后,模型对未来$T$个交易日股票
最低价平均值的预测效果较为不错,对未来投资计划具有一定的指导意义。


\section{不足与展望}

本次实验有诸多不足之处:
\begin{enumerate}
    \item 股票市场不仅受到历史数据的影响,还有其他方面的影响,其中有很多因素很难被量化,本次实验仅利用了股票市场的历史数据作为模型的输入变量,没有对这些指标进行进一步筛选和组合,在之后会进行更深一步的研究。
    \item 本次实验的重点是SVM回归模型的建立,没有与其他回归预测方法,如决策树、随机森林等进行对比,实验结果缺乏说服力,需要进一步进行研究。
    \item 目前混合核函数的研究是SVM模型优化的热点,但本文对核函数的选择原则仅是逐个测试,并从中选择了表现最好的RBF核函数,没有对各种混合核函数进行测试。此后的研究中,可以更多地尝试和了解混合核函数的适用范围。
    \item 另外,自动机器学习框架也是工业界的研究重点,其中超参数自动寻优的算法也层出不穷,本文仅对TPE算法进行了运用,没有与其他超参数寻优算法进行比较,对TPE算法的理论基础也没有进行很好的说明。
    \item 最后,本文对实验结论所给出的数据没有进一步进行分析,其包含的价值没有得到很好的体现。
\end{enumerate}

总而言之,伴随着中国经济实力的崛起,会有越来越多人开始关注股市的发展。
关于股票价格的预测问题,现有的预测手段虽然能为我们提供一定的投资建议,但是关于其背后的巨量的交易数据的价值,我们还所知甚少,任重而道远。
在未来,量化投资结合机器学习方法对股市进行预测一定会得到更完善的发展。